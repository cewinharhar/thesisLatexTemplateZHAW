%%%%%%%%%%%%%%%%%%%%%%%%%%%%%%%%%%%%%%%%%%%%%%%%%%%%%%%%%%%%%%%%%%%%%%%%%%%%%%%%%%
%   Vorlage Bericht AnCP-2
%   
%   Diese Vorlage beinhaltet die wichtigsten Packete und stellt ein einfaches
%   Layout für die Abgabe der Berichte im AnCP-2 zur Verfügung.
%   
%   Autor:      Thomas Keller (keeh@zhaw.ch)
%   Datum:      19.07.2019
%%%%%%%%%%%%%%%%%%%%%%%%%%%%%%%%%%%%%%%%%%%%%%%%%%%%%%%%%%%%%%%%%%%%%%%%%%%%%%%%%%

%%%%%%%%%%  Eingebundene Packete

\documentclass[a4paper,ngerman, twoside]{scrartcl}          %   Grunddefinition der Datei
\setkomafont{subsubsection}{\large}
\usepackage[utf8]{inputenc}                                 %   ermöglicht Umlaute direkt (ä,ö,ü,...)
\usepackage[ngerman]{babel}                                 %   Sprachpacket für Vordefinierte Blöcke
\usepackage{amssymb}                                        %   stellt eine Bibliothek von Sonderzeichen (ohne Mathe Umgebunghttps://www.overleaf.com/project/604d08ba921769b2534b87da zur verfügung
\usepackage{amsmath}                                        %   Diverse mathematische Funktionen
\usepackage[version=4]{mhchem}                              %   Chemische Formeln abbilden
\usepackage[hidelinks,pdftex]{hyperref}                                       %   Verlinkungen innerhalb des PDF
\usepackage{url}                                            %   Einstellung für Links
\usepackage[
			format=plain, 
			margin=10pt, 
			font=small,
			skip=10pt]{caption}                             %   Einstellung für Beschriftungen
\usepackage{pdfpages}                                       %   Einbinden externer PDF's
\usepackage[toc,page, title]{appendix}                      %   Diverse Einstellungen für den Anhang
\usepackage[
    backend=biber,      %   Einbindung mittels biber Backend
    style=chem-angew,   %   zitieren nach Chem. Angew. Journal
    biblabel=brackets,  %   Eckige Klammern
    maxnames=3,         %   Maximale Anzahl Autoren (vor et. al)
    minnames=3          %   Minimale Anzahl Autoren
    ]{biblatex}                                             %   Einbinden & Darstellen von Literatur
\usepackage{lscape}                                         %   einfaches Seitenlayout (quer)
\usepackage[onehalfspacing]{setspace}                       %   ändern von zeilenabständen
\usepackage{float}                                          %   positionierung von Gleitobjekten
\usepackage{fancybox}                                       %   verchiedene Container Formate
\usepackage[
	headsepline,
	footsepline,
	plainheadsepline,
	plainfootsepline,
	automark
]{scrlayer-scrpage}

\usepackage{longtable}
\usepackage{csquotes}
\raggedbottom

\usepackage[a4paper, 
			left=28mm, 
			right=22mm, 
			top=35mm, 
			bottom=30mm, 
			headheight=15mm, 
			footskip=15mm]{geometry}
\usepackage{tabularx}                                       %   erweiterte Tabelleneigenschaften
\usepackage{microtype}                                      %   hübsches Schriftbild
\usepackage{multirow}                                       %   Zeilen in Tabelle verbinden
\usepackage{booktabs}

\usepackage{subcaption}

\usepackage{rotating}

\usepackage{graphicx}

\usepackage{seqsplit}
\usepackage{ltablex}



%%%%%%%%%%  Einstellungen für die Packete
\urlstyle{rm}                                               %   setzt URL Schrift auf Std.
\setcounter{tocdepth}{2}                                    %   Darstellungstiefe des Inhaltsverzeichnis

% Anhang-Einstellungen
%\addto\captionsngerman{\let\appendixtocname\appendixname%
%\let\appendixpagename\appendixname}
\renewcommand{\appendixpagename}{Anhang} 
\renewcommand{\appendixtocname}{Anhang} 


% Literatureinstellungen
\addbibresource{referenzen.bib}
\DefineBibliographyStrings{ngerman}{                        % Schreibt et. al anstelle von und weitere
   andothers = {{et\,al\adddot}},            
}

% Groesseneinstellungen
\setlength{\evensidemargin}{0.0cm}
\setlength{\oddsidemargin}{0.0cm}
\setlength{\textwidth}{16cm}
\setlength{\topmargin}{-1cm}
\setlength{\textheight}{24cm}
\setlength{\parindent}{0.0cm}
\setlength{\headheight}{14pt} 

% Tabelleneinstellungen
\newcolumntype{L}[1]{>{\raggedright\arraybackslash}p{#1}} % linksbündig mit Breitenangabe
\newcolumntype{R}[1]{>{\raggedleft\arraybackslash}p{#1}} % rechtsbündig mit Breitenangabe
\newcolumntype{C}[1]{>{\centering\arraybackslash}m{#1}}

% neue Kommandos (Macro)
\newcommand{\bit}[1]{\emph{\textbf{#1}}}                    % bold italic \bit{asdf}
\newcommand{\atz}{\quad\quad}                               % links rechts leerzeichen Breite M \atz{asdf}
\newcommand{\bb}[1]{\textbf{#1}}                            % bold \bb{asdf}
\newcommand{\uu}[1]{\underline{#1}}                         % underline \uu{asdf}
\newcommand{\yo}[1]{\colorbox{Yellow}{#1}}

\pagestyle{scrheadings}	                                    % aktiviert kopf & fusszeile
\clearscrheadfoot		                                    % delets default header / footer
\renewcommand*{\headfont}{\normalfont}	                    % disable italic font

% Schriftbild
\setkomafont{sectioning}{\rmfamily\bfseries}   
% ändert Kapitelschrift in Serifenschrift

\newcommand{\bigline}{\rule{\linewidth}{1.5pt}}
\newcommand{\smallline}{\rule{.618\linewidth}{1pt}}
\usepackage{subcaption}

%%%%%%%%%%%%%%%%%%%%%%%%%%%%%%%%%%%%%%%%%%%%%%%%%%%%%%%%%%%%%%%%%%%%%%%%%%%%%%%%%%
%   Seitenlayouts
%%%%%%%%%%%%%%%%%%%%%%%%%%%%%%%%%%%%%%%%%%%%%%%%%%%%%%%%%%%%%%%%%%%%%%%%%%%%%%%%%%
\defpagestyle{Leer}{                % Leere kopf und Fusszeile
	(0pt, 0pt)
	{}{}{}
	(0pt, 0pt)
}
{	(0pt, 0pt)
	{}{}{}
	(0pt, 0pt)
}

\defpagestyle{Hauptteil}{           % Hauptteil
	(0pt, 0pt)
	{\leftmark\hfill\hfill}
	{\hfill\hfill\leftmark}
	{}
	(16cm, 1pt)
}
{	(16cm, 1pt)
	{\thepage\hfill\hfill}
	{\hfill\hfill\thepage}
	{}
	(0pt, 0pt)
}

\defpagestyle{Anhang}{              % Anhang
	(0pt, 0pt)
	{\text{Anhang}\hfill\hfill}
	{\hfill\hfill\text{Anhang}}
	{}
	(16cm, 1pt)
}
{	(16cm, 1pt)
	{\thepage\hfill\hfill}
	{\hfill\hfill\thepage}
	{}
	(0pt, 0pt)
}

\defpagestyle{Verzeichnisse}{   % Verzeichnisse
	(0pt, 0pt)
	{\leftmark\hfill\hfill}
	{\hfill\hfill\leftmark}
	{}
	(16cm, 1pt)
}
{	(16cm, 1pt)
	{\thepage\hfill\hfill}
	{\hfill\hfill\thepage}
	{}
	(0pt, 0pt)
}


%%%%%%%%%%%%%%%%%%%%%%%%%%%%%%%%%%%%%%%%%%%%%%%%%%%%%%%%%%%%%%%%%%%%%%%%%%%%%%%%%%
%   Dokument beginnt
%%%%%%%%%%%%%%%%%%%%%%%%%%%%%%%%%%%%%%%%%%%%%%%%%%%%%%%%%%%%%%%%%%%%%%%%%%%%%%%%%%

\begin{document}


%%%%%%%%%%%%%%%%    Titelseite
% Titelseite
\begin{titlepage}
\begin{table}
\begin{tabular}{L{300px}l}
\multirow{2}{220px}{\includegraphics[width=218px]{Bilder/zhaw.png}} & \\ \rule{0pt}{54px}
    & \textbf{\Large{Studiengang XXX}}\\
    & \textbf{\Large{Bachelorpraktikum}}\\
\end{tabular}
\end{table}
 
\vspace{10cm}

\vspace*{0.9cm}
\begin{center}
    \textbf{\Large{Bachelorarbeit}}\\ \vspace{1.0cm}
    \textbf{\huge{Titel Arbeit}}\\ \vspace{1.0cm}
    \textbf{\Large{Vertraulich}}
\end{center}

\vspace{7cm} 
\begin{tabular}{L{5cm}l}
\LARGE{Verfasser} &
Kevin Yar \\
\end{tabular}

\vspace{1cm}
\begin{tabular}{L{5cm}l}
	
\Large{Betreuer:} & Prof. Dr. Rebecca Buller \\ 
 & Dr. Sumire Honda \\
 & Dr. Michael Niklaus\\

\end{tabular}


\vspace{1.5cm}

\begin{tabular}{L{5cm}l}
\Large{Beginn:} & 15.02.2021 \\

\Large{Abgabe:} & 16.04.2021\\

% \Large{Abgabe:} & XX.XX.XXXX\\
\end{tabular}

\end{titlepage}


\leavevmode\thispagestyle{empty}\newpage

%%%%%%%%%%%%%%%%    Plagiatserklärung, Abstract, Inhalt
\pagestyle{Verzeichnisse}
\pagenumbering{Roman}
\setcounter{page}{1}

\input{1_Zus_Abstract.tex}
\leavevmode\thispagestyle{Verzeichnisse}%\newpage


\includegraphics[scale=0.25]{Bilder/zhaw.png}
\vspace{2cm}

\textsf{\textbf{\huge{Erklärung betreffend dem selbständigen Verfassen einer Arbeit im Departement Life Sciences und Facility Management}}}
	
\vspace{1cm}
	
Mit der Abgabe dieser Arbeit versichern die Studierenden, dass sie die Arbeit selbständig und ohne fremde Hilfe verfasst haben.

Die unterzeichnenden Studierenden erklären, dass alle verwendeten Quellen (auch Internetseiten) im Text oder Anhang korrekt ausgewiesen sind, d.h. dass die Arbeit keine Plagiate enthält, also keine Teile, die teilweise oder vollständig aus einem fremden Text oder einer fremden Arbeit unter Vorgabe der eigenen Urheberschaft bzw. ohne Quellenangabe übernommen worden sind.

Bei Verfehlungen aller Art treten der Paragraph 38*) (Unredlichkeit und Verfahren bei Unredlichkeit) der Studien- und Prüfungsordnung für die Bachelor-Studiengänge der Hochschule Wädenswil vom 1. September 2006 sowie die Bestimmungen der Disziplinarmassnahmen der Hochschulordnung in Kraft. 


\vspace{2.5cm}

\begin{table}[H]

\begin{tabular}{p{3cm}p{5cm}p{3cm}}
	Ort, Datum &  & Unterschrift \\
	&&\\
....................................	&&.........................................................\\
	&&\\
	....................................	&&.........................................................\\
	&&\\
\end{tabular}	
\end{table}


\vspace{2cm}



\begin{singlespace}
\textsf{\tiny{*)  38. Unredlichkeit
Bei Unredlichkeit gilt die Prüfung, Arbeit oder jede andere zu erbringende Leistung als nicht bestanden, eine Note  wird nicht erteilt. Unredlichkeiten können den Ausschluss von der Prüfung, die Ungültigerklärung eines Leistungsnachweises sowie die Verweigerung oder die Ungültigerklärung des Diploms zur Folge haben.
In der Regel ist die ganze Prüfung, Arbeit oder andere zu erbringende Leistung anlässlich des nächsten ordentlichen Termins zu wiederholen. Über Ausnahmen entscheidet die Leitung der Fachabteilung und der Prorektor oder die Prorektorin Lehre.}}
\end{singlespace}

%\newpage
\leavevmode\thispagestyle{Verzeichnisse}%\newpage

\tableofcontents
\newpage


%%%%%%%%%%%%%%%%    INHALT
\pagestyle{Hauptteil}


% % % ein Bild %%%%%%%%%%%%%%%%%

% \begin{figure}[H]
%     \centering
%     \includegraphics[width=.9\textwidth]{Bilder/Layout/zhaw.png}
%     \caption{}
%     \label{fig:0}
% \end{figure}


% % mehrere Bilder %%%%%%%%%%%%%

% \begin{figure}[H]
%     \centering
%     \begin{minipage}[t]{0.49\linewidth}
%     \centering
%     \includegraphics[width=\textwidth]{Bilder/Layout/zhaw.png}
%     \caption{}
%     \label{fig:1}
%     \end{minipage}
%     \hfill
%     \begin{minipage}[t]{0.49\linewidth}
%     \centering
%     \includegraphics[width=\textwidth]{Bilder/Layout/zhaw.png}
%     \caption{}
%     \label{fig:2}
%     \end{minipage}
% \end{figure}


% % Tabelle %%%%%%%%%%%%%%%%%%%%%

% \begin{table}[H]									
% \small \centering									
% \caption{}									
% \vspace{.3cm}									
%     \begin{tabular}{ccccc}									
% \hline									
% \textbf{}	&	\textbf{}	&	\textbf{}	&	\textbf{}	&	\textbf{}	\\ \hline
% 	&		&		&		&		\\ \hline
% 	&		&		&		&		\\ \hline
%     \end{tabular}									
%     \label{tab:}									
% \end{table}									


% % Tabelle über mehrere Seiten %%%

% \begin{longtable}{|p{5cm}|p{3cm}|p{3cm}|p{3cm}|}
% \caption{Übersicht aller verwendeten Chemikalien}\\
% \hline
% \textbf{Chemikalien} & \textbf{Hersteller} & \textbf{Sicherheit} & \textbf{H- und P-Sätze}   \\
% \hline
% \endfirsthead
% \multicolumn{4}{c}%
% {\tablename\ \thetable\ -- \textit{Fortsetzung von vorheriger Seite}} \\
% \hline
% \textbf{Chemikalien} & \textbf{Hersteller} & \textbf{Sicherheit} & \textbf{H- und P-Sätze}   \\
% \hline
% \endhead
% \hline \multicolumn{4}{r}{\textit{Fortsetzung auf der nächsten Seite}} \\
% \endfoot
% \hline
% \endlastfoot

% 1-(Boc-amino)-cyclohexancarboxylsäure\newline
% CAS-Nr.: 115951-16-1\newline
% M: 243.30 g/mol\newline
% 95 \%&Fluorochem&Reizend&H:\;302-312-315-319-332-335 \newline P:\;261-264-270-271302-313-337-313-363-403-233-501\\
% \hline

% 2-(7-Aza-1H-benzotriazole-1-yl9-1,1,3,3-tetramethyluronium (HATU)\newline
% CAS-Nr.: 148893-10-1\newline
% M: 30.24 g/mol\newline
% 98 \%&Fluorochem&Reizend&H:\;228-215-219-335  P:\;210-261-305-351-338\\
% \hline
   
% \end{longtable}


% % Mathematische Formeln %%%%%%

% \LaTeX{} kann selbstverständlich auch Gleichungen sauber nummeriert darstellen. Dafür muss das \$ Zeichen eingesetzt werden damit wechselt ihr in den Mathemodus und es ist Möglich im Text beispielsweise eine Definition zu machen $\lambda = \text{Wellenlänge}$. Alternativ können abgesetzte Formeln erzeugt werden. Das sieht dann so aus.
% \begin{equation}
%     y = a + bx
% \end{equation}

% Gleichungen lassen sich auch referenzieren...
% \begin{equation}
%     y = a + bx + cx^2
%     \label{Eq:Eq1}
% \end{equation}

% Refererenzieren geht auch mittels ref Befehl auf Gl. \ref{Eq:Eq1}. Es lassen sich auch Gleichungssysteme abbilden.

% \begin{eqnarray}
%     y & = & a + bx \label{eq:eq2}\\
%     y & = & a + bx + bx^2
% \end{eqnarray}

% davon ist grundsätzlich jede einzelne Gleichung zugänglich wie Gl. \ref{eq:eq2} zeigt. \newpage
\setcounter{page}{1}
\newpage
\addcontentsline{toc}{section}{Abkürzungsverzeichnis}
\section*{Abkürzungsverzeichnis}


\begin{table}[h]
\begin{tabular}{p{5cm} p{10cm} }
\toprule
bla & bli \\
\bottomrule
\end{tabular}
\end{table}
\leavevmode\thispagestyle{Hauptteil}\newpage


% \section*{Abkürzungsverzeichnis}


\begin{table}[h]
\begin{tabular}{p{5cm} p{10cm} }
\toprule
bla & bli \\
\bottomrule
\end{tabular}
\end{table} \newpage


\pagenumbering{arabic}

\input{1_Einleitung} \newpage
\input{2_Theorie} \newpage
\input{3_Versuchsdurchfuehrung} \newpage
\input{4_Ergebnisse} \newpage
\input{5_Diskussion} \newpage
\input{6_Ausblick} \newpage

%\input{referenzen} \newpage

%%%%%%%%%%%%%%%%    Literaturverzeichnis
\newpage
\addcontentsline{toc}{section}{Literatur}
\newpage
\printbibliography%[heading=none]
\leavevmode\thispagestyle{Hauptteil}\newpage
\pagestyle{Anhang}


%%%%%%%%%%%%%%%%    Anhang
\begin{appendix}
    

\listoffigures
\listoftables

\section{Anhang}

\subsection{Formelsammlung}

\begin{equation}
    \frac{| [R]-[S] |}{[R]+[S]} \cdot 100
\end{equation}

\end{appendix}

\end{document}